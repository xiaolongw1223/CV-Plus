%%%%%%%%%%%%%%%%%%%%%%%%%%%%%%%%%%%%%%%%%%%%%%%%%%%%%%%%%%%%%%%%%%%%%%%%%%%%%%%
% A clean template for an academic CV
%%%%%%%%%%%%%%%%%%%%%%%%%%%%%%%%%%%%%%%%%%%%%%%%%%%%%%%%%%%%%%%%%%%%%%%%%%%%%%%

\documentclass[11pt, a4paper]{article}

% Identifying information
\newcommand{\dept}{Department of Earth \& Atmospheric Sciences}
\newcommand{\university}{University of Houston}
\newcommand{\Title}{Curriculum Vit\ae}
\newcommand{\FirstName}{Xiaolong}
\newcommand{\LastName}{Wei}
\newcommand{\Initials}{X}
\newcommand{\MyName}{\FirstName\ \LastName}
\newcommand{\Wei}{\textbf{\LastName, \Initials.}}  % For citations
\newcommand{\WeiSun}{\textbf{\LastName, \Initials.} and Sun, J.}  % For citations
\newcommand{\Email}{xiaolongw1223@gmail.com}
\newcommand{\Website}{researchgate.net/profile/Xiaolong\_Wei}
\newcommand{\Lab}{https://sites.google.com/view/jiajiasun}
\newcommand{\ORCID}{0000-0002-3160-6086}
\newcommand{\Affiliation}{\dept, \university}
\newcommand{\Address}{
  Room 126, Science \& Research Building 1, 3507 Cullen Blvd, Houston, Texas, USA
}

% Control the font size
\usepackage{anyfontsize}

% For fancy and multipage tables
\usepackage{tabularx}
\usepackage{ltablex}

% Define a new environment to place all CV entries in a 2-column table.
% Left column are the dates, right column the entries.
\usepackage{environ}
\NewEnviron{EntriesTable}{
	\TablePad
	\begin{tabularx}{\textwidth}{@{}p{0.15\textwidth}@{\hspace{0.02\textwidth}}p{0.88\textwidth}@{}}
		\BODY
	\end{tabularx}
}

% Macros
\newcommand{\Duration}[2]{\fontsize{10pt}{0}\selectfont #1--#2}
\newcommand{\Year}[1]{\fontsize{10pt}{0}\selectfont #1}
\newcommand{\Future}{future}
\newcommand{\Review}{under review}
\newcommand{\Revision}{under revision}
\newcommand{\Appointment}[4]{\textbf{#1} \newline #2 \newline #3 \newline #4}
\newcommand{\DOI}[1]{doi:\href{https://doi.org/#1}{#1}}

% Define command to insert month name and year as date
\usepackage{datetime}
\newdateformat{monthyear}{\monthname[\THEMONTH], \THEYEAR}

% Set the page margins
\usepackage[left=1in,right=1in,top=1in,bottom=1in]{geometry}

% Increase the line spacing
\renewcommand{\baselinestretch}{1}
% and the spacing between rows in tables
\renewcommand{\arraystretch}{1.5}

% Remove space between items in itemize and enumerate
\usepackage{enumitem}
\setlist{itemsep=0.2em}

% numbering items in reversed orders
\usepackage{etaremune}[start=1]

% Use custom colors
\usepackage[usenames,dvipsnames]{xcolor}

% Set fonts. Requires compilation with xelatex
\usepackage{fontspec}  % required to make older xelatex compile with UTF8

% Set fancy headers
\usepackage{fancyhdr}
\pagestyle{fancy}
\fancyhf{}
\chead{
  \fontsize{10pt}{12pt}\selectfont
  \MyName
  \hspace{0.2cm} -- \hspace{0.2cm}
  \Title
  \hspace{0.2cm} -- \hspace{0.2cm}
  \monthyear\today
}
\rhead{}
\cfoot{\fontsize{10pt}{0}\selectfont \thepage}
\renewcommand{\headrulewidth}{0pt}

% Metadata for the PDF output and control of hyperlinks
\usepackage[colorlinks=true]{hyperref}
\hypersetup{
  pdftitle={\MyName\ - \Title},
  pdfauthor={\MyName},
  linkcolor=blue,
  citecolor=blue,
  filecolor=black,
  urlcolor=MidnightBlue
}


\begin{document}


% No header for the first page
\thispagestyle{empty}


%%%%%%%%%%%%%%%%%%%%%%%%%%%%%%%%%%%%%%%%%%%%%%%%%%%%%%%%%%%%%%%%%%%%%%%%%%%%%%%
% HEADER
\begin{center}

	{\fontsize{36pt}{0}\selectfont \MyName}
	\\[-0.1cm]
	\rule{\textwidth}{0.6pt}
	\\[0.4cm]
	{\fontsize{10pt}{0}\selectfont
		\Affiliation
		\\[0.1cm]
		\Address
		\\[0.1cm]
		Email: \href{mailto:\Email}{\Email}
		\, | \,
		ORCID: \href{https://orcid.org/\ORCID}{\ORCID}
		\\
		Website: \href{https://www.\Website}{\Website}
	}

\end{center}


%%%%%%%%%%%%%%%%%%%%%%%%%%%%%%%%%%%%%%%%%%%%%%%%%%%%%%%%%%%%%%%%%%%%%%%%%%%%%%%
\section*{Education}
\begin{EntriesTable}

  \Duration{2018}{Present}  &
  \textbf{Ph.D. in Geophysics}, University of Houston, Houston, USA
  \\
  \Duration{2015}{2018}  &
  \textbf{M.S. in Geology}, Northwest University, Xi'an, China
  \\
  \Duration{2011}{2015}  &
  \textbf{B.S. in Geophysics}, China University of Geosciences, Beijing, China

\end{EntriesTable}


%%%%%%%%%%%%%%%%%%%%%%%%%%%%%%%%%%%%%%%%%%%%%%%%%%%%%%%%%%%%%%%%%%%%%%%%%%%%%%%
\section*{Research Interests}
\begin{itemize}

	\item Geophysical separate and joint inversions
	\item Uncertainty analysis of determinsitc and stochastic inverse problems
	\item Geology differentiation and natural resources explorations
	\item Deep learning algorithms applied to geophysical and/or geological interpretations

\end{itemize}


%%%%%%%%%%%%%%%%%%%%%%%%%%%%%%%%%%%%%%%%%%%%%%%%%%%%%%%%%%%%%%%%%%%%%%%%%%%%%%%
\section*{Awards \& Honors}
\begin{EntriesTable}
  \Year{2022}  &
	The Innovation Prize in Frank Arnott - Next Generation Explorers Award (\$3,000CAD)
  	\\
	\Year{2022}  &
	SEG Lucien LaCoste Scholarship (\$5,305.12)
	\\
	\Year{2022}  &
	The Best Paper in the Mining Sessions at 2021 IMAGE Annual Meeting, Denver, CO, USA (co-author)
	\\
	\Year{2022}  &
	The Best Student Paper in the Mining Sessions at 2021 IMAGE Annual Meeting, Denver, CO, USA
	\\
	\Year{2021}  &
	Student Travel Award, University of Houston, Houston, USA
	\\
	\Year{2021}  &
	Student Research Funding (paid directly to student), University of Houston, Houston, USA (\$1,000)
	\\
	\Year{2021}  &
	SEG Technical Program Registration Grant
	\\
	\Year{2021}  &
	SEG John R. Butler Jr. Scholarship (\$510.86)
	\\
	\Year{2021} &
	The Best Poster in the Mining Sessions at 2020 SEG Annual Meeting, Online
	\\
	\Duration{2020}{2021}  &
	Outstanding Academic Achievement, University of Houston, Houston, USA (\$700\times2)
	\\
	\Duration{2016}{2018}  &
	The First Prize Scholarship, Northwest University, Xi'an, China
	\\
	\Year{2015}  &
	The Best Bachelor Thesis, China University of Geosciences, Beijing, China
	\\
	\Year{2013}  &
	The Second Prize Scholarship, China University of Geosciences, Beijing, China

\end{EntriesTable}


%%%%%%%%%%%%%%%%%%%%%%%%%%%%%%%%%%%%%%%%%%%%%%%%%%%%%%%%%%%%%%%%%%%%%%%%%%%%%%%
\section*{Publications}
\subsection*{Peer-reviewed}
\begin{etaremune}

	\item
	\Wei, Li, K. and Sun, J., 2021 Mapping critical mineral resources using airborne geophysics, 3D joint inversion and geology differentiation: A case study of a buried niobium deposit in the Elk Creek carbonatite, Nebraska, USA. \emph{Geophysical Prospecting}. \Review

	\item
	\WeiSun, 2021. 3D probabilistic geology differentiation based on airborne geophysics, mixed Lp norm joint inversion and petrophysical measurements. \emph{Geophysics}. \Review

	\item
	Hu, Y., \Wei, Wu, X., Sun, J., Chen, J., Huang, Y. and Chen, J., 2021. A deep learning enhanced framework for multi-physics joint inversion. \emph{Geophysics}. \Revision

	\item
	\WeiSun, 2021. Uncertainty analysis of 3D potential-field deterministic inversion using mixed L p norms. \emph{Geophysics}, 86(6), pp.G133-G158.
	\DOI{10.1190/geo2020-0672.1}

	\item
	Sun, J. and \Wei, 2020. Recovering sparse models in 3D potential‐field inversion without bound dependence or staircasing problems using a mixed Lp‐norm regularization. \emph{Geophysical Prospecting}, 69(4), pp.901-910.
	\DOI{10.1111/1365-2478.13063}.

	\item
	Sun, J., Melo, A., Kim, J.D. and \Wei, 2020. Unveiling the 3D undercover structure of a Precambrian intrusive complex by integrating airborne magnetic and gravity gradient data into 3D quasi-geology model building. \emph{Interpretation}, 8(4), pp.1-50.
	\DOI{10.1190/INT-2019-0273.1}.

\end{etaremune}

%\subsection*{In preparation}
%\begin{etaremune}
%	\item
%	Salt body imaging
%
%	\item
%	Antarctic airborne geophysics
%
%\end{etaremune}

\subsection*{Conference proceedings}
\begin{etaremune}

	\item
	\WeiSun, 2021. 3D probabilistic geology differentiation using mixed L p norm joint inversion constrained by petrophysical information. In \emph{IMAGE Technical Program Expanded Abstracts 2021} \DOI{10.1190/segam2021-3586619.1}.

	\item
	\WeiSun, 2021. Uncertainty analysis of 3D geophysical inversion using airborne gravity gradient data conditioned on rock sample measurements. In \emph{IMAGE Technical Program Expanded Abstracts 2021} \DOI{10.1190/segam2021-3586552.1}.

	\item
	Hu, Y., \Wei, Wu, X., Sun, J., Chen, J., Chen, J., Huang, Y., 2021. Deep learning-enhanced multiphysics joint inversion. In \emph{IMAGE Technical Program Expanded Abstracts 2021} \DOI{10.1190/segam2021-3583667.1}.

	\item
	Li, K., \Wei, Sun, J., 2021. Geophysical characterization of a buried niobium and rare earth element deposit using 3D joint inversion and geology differentiation: A case study on the Elk Creek carbonatite2021. In \emph{IMAGE Technical Program Expanded Abstracts 2021} \DOI{10.1190/segam2021-3585069.1}.

	\item
	\WeiSun, 2020. Uncertainty analysis of joint inversion using mixed Lp-norm regularization. In \emph{SEG Technical Program Expanded Abstracts 2020} (pp. 925-929). Society of Exploration Geophysicists. \DOI{10.1190/segam2020-3428359.1}.

	\item
	\WeiSun, 2020. Quantifying uncertainties of deterministic geophysical inversions using mixed Lp norms. In \emph{SEG Technical Program Expanded Abstracts 2020} (pp. 1404-1408). Society of Exploration Geophysicists. \DOI{10.1190/segam2020-3420227.1}.

	\item
	Sun, J., Melo, A., Deok Kim, J. and \Wei, 2020. Characterizing a Precambrian intrusive complex by integrating potential field data into 3D quasi-geology model building. In \emph{SEG Technical Program Expanded Abstracts 2020} (pp. 1374-1378). Society of Exploration Geophysicists. \DOI{10.1190/segam2020-3428385.1}.

\end{etaremune}


\subsection*{Conference abstracts}
\begin{etaremune}

	\item
	\WeiSun, 2021, December. Building 3D probabilistic geology differentiation models using mixed Lp norm joint inversion, airborne geophysics and petrophysical information. In \emph{AGU Fall Meeting Abstracts}.

	\item
	\WeiSun, 2021, December. Analyzing uncertainty of 3D inversion using airborne geophysical data conditioned on petrophysical measurements. In \emph{AGU Fall Meeting Abstracts}.

	\item
	Li, K., \Wei, Sun, J., 2021, December. Characterizing a buried niobium deposit using airborne geophysics, joint inversion, and geology differentiation. In \emph{AGU Fall Meeting Abstracts}.

\end{etaremune}


\subsection*{Open code and data}
\begin{etaremune}

	\item
	\WeiSun, 2021. Joint inversion of gravity gradient and magnetic data using mixed Lp norm regularization (1.0). \emph{Zenodo}. \DOI{10.5281/zenodo.5774303}.

	\item
	\WeiSun, 2021. Interactive geology differentiation and 3D visualization of geological units (1.0). \emph{Zenodo}. \DOI{10.5281/zenodo.5774309}.

	\item
	Sun, J., and \Wei, 2020. Solving the bound dependence and staircasing problems in 3D potential-field sparse inversions using a mixed Lp-norm regularization (1.0). \emph{Zenodo}. \DOI{10.5281/zenodo.4057134}.

\end{etaremune}



%%%%%%%%%%%%%%%%%%%%%%%%%%%%%%%%%%%%%%%%%%%%%%%%%%%%%%%%%%%%%%%%%%%%%%%%%%%%%%%
\section*{Professional Service \& Outreach}

\subsection*{Peer-Reviewer}

\begin{EntriesTable}
	\Duration{2022}{present}  &
	Geocarto International, SEG Conference Proceeding
	\\
	\Duration{2021}{present}  &
	Geophysics, Geophysical Journal International, IEEE Transactions on Geoscience and Remote Sensing, Acta Geophysica
\end{EntriesTable}

\subsection*{Conferences}

\begin{EntriesTable}
	\Year{2022} &
	Session Chair for GM 1: Inversion Insights at IMAGE Annual Meeting, Houston, Texas, USA
	\\
	\Year{2021} &
	Session Chair for MG P1: New Methods and Case Histories 1 at IMAGE Annual Meeting (SEG and AAPG joint annual conference), Denver, Colorado, USA
\end{EntriesTable}

\subsection*{Affiliations}

\begin{EntriesTable}
	\Duration{2022}{Present}  &
	European Geosciences Union (EGU)
	\\
	\Duration{2021}{Present}  &
	Geophysical Society of Houston (GSH)
	\\
	\Duration{2020}{Present}  &
	American Geophysical Union (AGU), European Association of Geoscientists \& Engineers (EAGE)
	\\
	\Duration{2018}{Present}  &
	Society of Exploration Geophysicists (SEG)
\end{EntriesTable}

\subsection*{Others}
\begin{EntriesTable}
	\Duration{2020}{2021}  &
	Contributor of the joint inversion code in SimPEG (https://simpeg.xyz/)
\end{EntriesTable}

%%%%%%%%%%%%%%%%%%%%%%%%%%%%%%%%%%%%%%%%%%%%%%%%%%%%%%%%%%%%%%%%%%%%%%%%%%%%%%%

\section*{Teaching Experience}
\begin{EntriesTable}

	\Year{2020}  &
	GEOL7330: Potential Field Methods of Geophysical Exploration (graduate core course), \textbf{guest lecturer}. \emph{University of Houston}.
	\\
	\Year{2019}  &
	GEOL4355: Geophysical Field Camp, \textbf{teaching assistant}. \emph{University of Houston}.

\end{EntriesTable}


%%%%%%%%%%%%%%%%%%%%%%%%%%%%%%%%%%%%%%%%%%%%%%%%%%%%%%%%%%%%%%%%%%%%%%%%%%%%%%%
\section*{Invited Talks}
\begin{EntriesTable}

	\Year{011/2021} &
	\WeiSun\ Build probabilistic quasi-geology models based on multiple airborne geophysical data and sparse joint inversions (online). \emph{Geophysical Society of Houston}.
	\\

	\Year{09/2021} &
	\WeiSun\ From deterministic to probabilistic geoscience modeling: analyzing uncertainties of geophysical inversions and constructing probabilistic subsurface models conditioned on petrophysical measurements (online). \emph{SimPEG monthly seminar}.

\end{EntriesTable}
%%%%%%%%%%%%%%%%%%%%%%%%%%%%%%%%%%%%%%%%%%%%%%%%%%%%%%%%%%%%%%%%%%%%%%%%%%%%%%%
\section*{Certifications}
\begin{EntriesTable}

	\Year{2022} &
	Remote pilot for the small unmanned aircraft system issued by Federal Aviation Administration
	\\
	\Year{2021} &
	FAA Part 107 Knowledge Test Prep for Drone Pilot on Udemy, Inc.
	\\
	\Year{2021} &
	ISInProG@Lario - 2021 International School on Inverse Problems in Geophysics on the shore of the Lario Lake
	\\
	\Year{2021} &
	Magnetotellurics (MT) short course given by Dr. Alan G. Jones
	\\
	\Year{2018}{}  &
	Machine Learning course given by Dr. Andrew Ng on Coursera, Inc.

\end{EntriesTable}

\end{document}
