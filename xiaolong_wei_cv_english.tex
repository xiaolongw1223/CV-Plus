%%%%%%%%%%%%%%%%%%%%%%%%%%%%%%%%%%%%%%%%%%%%%%%%%%%%%%%%%%%%%%%%%%%%%%%%%%%%%%%
% A clean template for an academic CV
%%%%%%%%%%%%%%%%%%%%%%%%%%%%%%%%%%%%%%%%%%%%%%%%%%%%%%%%%%%%%%%%%%%%%%%%%%%%%%%

\documentclass[11pt, a4paper]{article}

% Identifying information
\newcommand{\dept}{Department of Earth \& Atmospheric Sciences}
\newcommand{\university}{University of Houston}
\newcommand{\Title}{Curriculum Vit\ae}
\newcommand{\FirstName}{Xiaolong}
\newcommand{\LastName}{Wei}
\newcommand{\Initials}{X}
\newcommand{\MyName}{\FirstName\ \LastName}
\newcommand{\Wei}{\textbf{\LastName, \Initials.}}  % For citations
\newcommand{\WeiSun}{\textbf{\LastName, \Initials.} and Sun, J.}  % For citations
\newcommand{\Email}{xiaolongw1223@gmail.com}
\newcommand{\Website}{researchgate.net/profile/Xiaolong\_Wei}
\newcommand{\Lab}{https://sites.google.com/view/jiajiasun}
\newcommand{\ORCID}{0000-0002-3160-6086}
\newcommand{\Affiliation}{\dept, \university}
\newcommand{\Address}{
  Room 126, Science \& Research Building 1, 3507 Cullen Blvd, Houston, Texas, USA
}

% Control the font size
\usepackage{anyfontsize}

% For fancy and multipage tables
\usepackage{tabularx}
\usepackage{ltablex}

% Define a new environment to place all CV entries in a 2-column table.
% Left column are the dates, right column the entries.
\usepackage{environ}
\NewEnviron{EntriesTable}{
	\TablePad
	\begin{tabularx}{\textwidth}{@{}p{0.15\textwidth}@{\hspace{0.02\textwidth}}p{0.88\textwidth}@{}}
		\BODY
	\end{tabularx}
}

% Macros
\newcommand{\Duration}[2]{\fontsize{10pt}{0}\selectfont #1--#2}
\newcommand{\Year}[1]{\fontsize{10pt}{0}\selectfont #1}
\newcommand{\Future}{future}
\newcommand{\Review}{under review}
\newcommand{\Revision}{under revision}
\newcommand{\Appointment}[4]{\textbf{#1} \newline #2 \newline #3 \newline #4}
\newcommand{\DOI}[1]{doi:\href{https://doi.org/#1}{#1}}

% Define command to insert month name and year as date
\usepackage{datetime}
\newdateformat{monthyear}{\monthname[\THEMONTH], \THEYEAR}

% Set the page margins
\usepackage[left=1in,right=1in,top=1in,bottom=1in]{geometry}

% Increase the line spacing
\renewcommand{\baselinestretch}{1.1}
% and the spacing between rows in tables
\renewcommand{\arraystretch}{1.5}

% Remove space between items in itemize and enumerate
\usepackage{enumitem}
\setlist{itemsep=0.2em}

% numbering items in reversed orders
\usepackage{etaremune}[start=1]

% Use custom colors
\usepackage[usenames,dvipsnames]{xcolor}

% Set fonts. Requires compilation with xelatex
\usepackage{fontspec}  % required to make older xelatex compile with UTF8

% Set fancy headers
\usepackage{fancyhdr}
\pagestyle{fancy}
\fancyhf{}
\chead{
  \fontsize{10pt}{12pt}\selectfont
  \MyName
  \hspace{0.2cm} -- \hspace{0.2cm}
  \Title
  \hspace{0.2cm} -- \hspace{0.2cm}
  \monthyear\today
}
\rhead{}
\cfoot{\fontsize{10pt}{0}\selectfont \thepage}
\renewcommand{\headrulewidth}{0pt}

% Metadata for the PDF output and control of hyperlinks
\usepackage[colorlinks=true]{hyperref}
\hypersetup{
  pdftitle={\MyName\ - \Title},
  pdfauthor={\MyName},
  linkcolor=blue,
  citecolor=blue,
  filecolor=black,
  urlcolor=MidnightBlue
}


\begin{document}


% No header for the first page
\thispagestyle{empty}


%%%%%%%%%%%%%%%%%%%%%%%%%%%%%%%%%%%%%%%%%%%%%%%%%%%%%%%%%%%%%%%%%%%%%%%%%%%%%%%
% HEADER
\begin{center}
	
	{\fontsize{36pt}{0}\selectfont \MyName}
	\\[-0.1cm]
	\rule{\textwidth}{0.6pt}
	\\[0.4cm]
	{\fontsize{10pt}{0}\selectfont
		\Affiliation
		\\[0.1cm]
		\Address
		\\[0.1cm]
		Email: \href{mailto:\Email}{\Email}
		\, | \,
		ORCID: \href{https://orcid.org/\ORCID}{\ORCID}
		\\
		Website: \href{https://www.\Website}{\Website}
	}

\end{center}


%%%%%%%%%%%%%%%%%%%%%%%%%%%%%%%%%%%%%%%%%%%%%%%%%%%%%%%%%%%%%%%%%%%%%%%%%%%%%%%
\section*{Education}
\begin{EntriesTable}
	
  \Duration{2018}{Present}  &
  \textbf{Ph.D in Geophysics}, University of Houston, Houston, USA
  \\
  \Duration{2015}{2018}  &
  \textbf{M.S. in Geology}, Northwest University, Xi'an, China
  \\
  \Duration{2011}{2015}  &
  \textbf{B.S. in Geophysics}, China University of Geosciences, Beijing, China
  
\end{EntriesTable}


%%%%%%%%%%%%%%%%%%%%%%%%%%%%%%%%%%%%%%%%%%%%%%%%%%%%%%%%%%%%%%%%%%%%%%%%%%%%%%%
\section*{Research Interests}
\begin{itemize}
	
	\item Geophysical inverse problems for multiple data sets (e.g., gravity, gravity gradiometry and magnetic)
	\item Structural similarity constraint joint inversion
	\item Uncertainty analysis in geophysical separate/joint inversions in both deterministic and stochastic frameworks
	\item Geology differentiation models
	\item Machine/deep learning algorithms applied to geophysical data interpretations
	
\end{itemize}


%%%%%%%%%%%%%%%%%%%%%%%%%%%%%%%%%%%%%%%%%%%%%%%%%%%%%%%%%%%%%%%%%%%%%%%%%%%%%%%
\section*{Awards \& Honors}
\begin{EntriesTable}
	
	\Year{2021}  &
	SEG Technical Program Registration Grant
	\\
	\Year{2021}  &
	John R. Butler Jr. Scholarship from SEG
	\\
	\Year{2021} &
	The Best Poster in the Mining Sessions at the 2020 SEG Annual Meeting
	\\
	\Duration{2020}{2021}  &
	Outstanding Academic Achievement, University of Houston, Houston, USA (\times2)
	\\
	\Duration{2016}{2018}  &
	The First Prize Scholarship, Northwest University, Xi'an, China (\times3)
	\\
	\Year{2015}  &
	The Best Bachelor Thesis, China University of Geosciences, Beijing, China
	\\
	\Year{2013}  &
	The Second Prize Scholarship, China University of Geosciences, Beijing, China
	\\
	\Year{2012}  &
	Outstanding Volunteer for rural elementary schools, China University of Geosciences, Beijing, China	
	
\end{EntriesTable}


%%%%%%%%%%%%%%%%%%%%%%%%%%%%%%%%%%%%%%%%%%%%%%%%%%%%%%%%%%%%%%%%%%%%%%%%%%%%%%%
\section*{Publications}
\subsection*{Peer-reviewed}
\begin{etaremune}
	
	\item 
	Hu, Y., \Wei, Wu, X., Sun, J., Chen, J., Huang, Y., Chen, J., 2021. A deep learning enhanced framework for multi-physics joint inversion. \emph{Geophysics}. \Review
	
	\item
	\WeiSun, 2021. Uncertainty analysis of 3D potential-field deterministic inversion using mixed L p norms. \emph{Geophysics}, 86(6), pp.1-103.

	\item
	Sun, J., \Wei, 2020. Recovering sparse models in 3D potential‐field inversion without bound dependence or staircasing problems using a mixed Lp‐norm regularization. \emph{Geophysical Prospecting}.
	\DOI{10.1111/1365-2478.13063}.

	\item
	Sun, J., Melo, A., Kim, J.D. and \Wei, 2020. Unveiling the 3D undercover structure of a Precambrian intrusive complex by integrating airborne magnetic and gravity gradient data into 3D quasi-geology model building. \emph{Interpretation}, 8(4), pp.1-50.
	\DOI{10.1190/INT-2019-0273.1}.

\end{etaremune}

\subsection*{In preparation}
\begin{etaremune}
	
	\item
	\WeiSun, 2021. Uncertainty analysis of 3D geology differentiation models via joint inversion. 
	
	\item 
	Li, K., \Wei, Sun, J., 2021 Mapping critical mineral resources using airborne geophysics, 3D joint inversion and geology differentiation: A case study of a buried niobium deposit in the Elk Creek carbonatite, Nebraska, USA
	
\end{etaremune}

\subsection*{Conference proceeding}
\begin{etaremune}
	
	\item 
	\WeiSun, 2021. 3D probabilistic geology differentiation using mixed L p norm joint inversion constrained by petrophysical information. In \emph{IMAGE Technical Program Expanded Abstracts 2021} \DOI{10.1190/segam2021-3586619.1}.

	\item 
	\WeiSun, 2021. Uncertainty analysis of 3D geophysical inversion using airborne gravity gradient data conditioned on rock sample measurements. In \emph{IMAGE Technical Program Expanded Abstracts 2021} \DOI{10.1190/segam2021-3586552.1}.
	
	\item 
	Hu, Y., \Wei, Wu, X., Sun, J., Chen, J., Chen, J., Huang, Y., 2021. Deep learning-enhanced multiphysics joint inversion. In \emph{IMAGE Technical Program Expanded Abstracts 2021} \DOI{10.1190/segam2021-3583667.1}.
	
	\item 
	Li, K., \Wei, Sun, J., 2021. Geophysical characterization of a buried niobium and rare earth element deposit using 3D joint inversion and geology differentiation: A case study on the Elk Creek carbonatite2021. In \emph{IMAGE Technical Program Expanded Abstracts 2021} \DOI{10.1190/segam2021-3585069.1}.

	\item 
	\WeiSun, 2020. Uncertainty analysis of joint inversion using mixed Lp-norm regularization. In \emph{SEG Technical Program Expanded Abstracts 2020} (pp. 925-929). Society of Exploration Geophysicists. \DOI{10.1190/segam2020-3428359.1}.

	\item 
	\WeiSun, 2020. Quantifying uncertainties of deterministic geophysical inversions using mixed Lp norms. In \emph{SEG Technical Program Expanded Abstracts 2020} (pp. 1404-1408). Society of Exploration Geophysicists. \DOI{10.1190/segam2020-3420227.1}.
	
	\item 
	Sun, J., Melo, A., Deok Kim, J. and \Wei, 2020. Characterizing a Precambrian intrusive complex by integrating potential field data into 3D quasi-geology model building. In \emph{SEG Technical Program Expanded Abstracts 2020} (pp. 1374-1378). Society of Exploration Geophysicists. \DOI{10.1190/segam2020-3428385.1}.
		
\end{etaremune}


%%%%%%%%%%%%%%%%%%%%%%%%%%%%%%%%%%%%%%%%%%%%%%%%%%%%%%%%%%%%%%%%%%%%%%%%%%%%%%%
%\section*{Invited Talks}
%\begin{etaremune}
%	
%	\item
%	\Wei, April 2020. Characterizing a Precambrian intrusive complex by integrating potential field data into 3D quasi-geology model building. Society of Exploration Geophysicists. 
%	\Future
%	
%\end{etaremune}

%%%%%%%%%%%%%%%%%%%%%%%%%%%%%%%%%%%%%%%%%%%%%%%%%%%%%%%%%%%%%%%%%%%%%%%%%%%%%%%
\section*{Reviewers}
\begin{EntriesTable}
	
	\Duration{2021}{present}  &
	Acta Geophysica
	\\
	\Duration{2021}{present}  &
	IEEE Transactions on Geoscience and Remote Sensing

\end{EntriesTable}



%%%%%%%%%%%%%%%%%%%%%%%%%%%%%%%%%%%%%%%%%%%%%%%%%%%%%%%%%%%%%%%%%%%%%%%%%%%%%%%
\section*{Professional Affiliations \& Activities}
\begin{EntriesTable}
	
	\Year{08/2021} &
	Participant of 2021 ISInProG@Lario - 2021 International School on Inverse Problems in Geophysics on the shore of the Lario Lake
	\\
	\Duration{2020}{}  &
	Contributor of joint inversion code in SimPEG (https://simpeg.xyz/)
	\\
	\Duration{2020}{Present}  &
	American Geophysical Union (AGU)	
	\\
	\Duration{2020}{Present}  &
	European Association of Geoscientists \& Engineers (EAGE)
	\\
	\Duration{2018}{Present}  &
	Society of Exploration Geophysicists (SEG) 
	
\end{EntriesTable}



%%%%%%%%%%%%%%%%%%%%%%%%%%%%%%%%%%%%%%%%%%%%%%%%%%%%%%%%%%%%%%%%%%%%%%%%%%%%%%%

\section*{Teaching Experiences}
\begin{EntriesTable}
	
	\Year{2020}  &
	GEOL7330: Potential Field Methods of Geophysical Exploration (graduate core course), \textbf{guest lecturer}, University of Houston	
	\\
	\Year{2019}  &
	GEOL4355: Geophysical Field Camp, \textbf{teaching assistant}, University of Houston	
	
\end{EntriesTable}


%%%%%%%%%%%%%%%%%%%%%%%%%%%%%%%%%%%%%%%%%%%%%%%%%%%%%%%%%%%%%%%%%%%%%%%%%%%%%%%
\section*{Invited Talks}
\begin{EntriesTable}
	\Year{09/2021} &
	\WeiSun, From deterministic to probabilistic geoscience modeling: analyzing uncertainties of geophysical inversions and constructing probabilistic subsurface models conditioned on petrophysical measurements, SimPEG monthly seminar. 

\end{EntriesTable}
%%%%%%%%%%%%%%%%%%%%%%%%%%%%%%%%%%%%%%%%%%%%%%%%%%%%%%%%%%%%%%%%%%%%%%%%%%%%%%%
\section*{Certifications}
\begin{EntriesTable}
	\year{2021} &
	Magnetotellurics (MT) short course given by Dr. Alan G. Jones.
	\\
	\Year{2018}{}  &
	Certificate signed by Prof. Andrew Ng upon successfully completing the online
	machine learning course provided by Stanford University through Coursera Inc.
	
\end{EntriesTable}

\end{document}
