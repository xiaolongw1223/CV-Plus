%%%%%%%%%%%%%%%%%%%%%%%%%%%%%%%%%%%%%%%%%%%%%%%%%%%%%%%%%%%%%%%%%%%%%%%%%%%%%%%
% A clean template for an academic CV
%%%%%%%%%%%%%%%%%%%%%%%%%%%%%%%%%%%%%%%%%%%%%%%%%%%%%%%%%%%%%%%%%%%%%%%%%%%%%%%

\documentclass[11pt, a4paper]{article}

% Identifying information
\newcommand{\dept}{Department of Earth \& Planetary Sciences}
\newcommand{\university}{Stanford University}
\newcommand{\Title}{Curriculum Vit\ae}
\newcommand{\FirstName}{Xiaolong}
\newcommand{\LastName}{Wei}
\newcommand{\Initials}{X}
\newcommand{\MyName}{\FirstName\ \LastName}
\newcommand{\Wei}{\textbf{\LastName, \Initials.}}  % For citations
\newcommand{\WeiSun}{\textbf{\LastName, \Initials.} and Sun, J.}  % For citations
\newcommand{\Email}{xwei2@stanford.edu}
\newcommand{\Website}{https://xiaolongw1223.github.io/}
\newcommand{\Google}{https://scholar.google.com/citations?user=TyBgOgIAAAAJ&hl=en}
\newcommand{\Lab}{https://sites.google.com/view/jiajiasun}
\newcommand{\ORCID}{https://orcid.org/0000-0002-3160-6086}
\newcommand{\Github}{https://github.com/xiaolongw1223}
\newcommand{\Affiliation}{\dept, \university}
\newcommand{\Address}{
	Room 315, Green Building, 367 Panama St, Stanford,  California,  94305
}

% Control the font size
\usepackage{anyfontsize}

% For fancy and multipage tables
\usepackage{tabularx}
\usepackage{ltablex}

% Define a new environment to place all CV entries in a 2-column table.
% Left column are the dates, right column the entries.
\usepackage{environ}
\NewEnviron{EntriesTable}{
	\TablePad
	\begin{tabularx}{\textwidth}{@{}p{0.15\textwidth}@{\hspace{0.02\textwidth}}p{0.88\textwidth}@{}}
		\BODY
	\end{tabularx}
}

% Macros
\newcommand{\Duration}[2]{\fontsize{10pt}{0}\selectfont #1--#2}
\newcommand{\Year}[1]{\fontsize{10pt}{0}\selectfont #1}
\newcommand{\Future}{future}
\newcommand{\Review}{under review}
\newcommand{\Revision}{under revision}
\newcommand{\Appointment}[4]{\textbf{#1} \newline #2 \newline #3 \newline #4}
\newcommand{\DOI}[1]{doi:\href{https://doi.org/#1}{#1}}

% Define command to insert month name and year as date
\usepackage{datetime}
\newdateformat{monthyear}{\monthname[\THEMONTH], \THEYEAR}

% Set the page margins
\usepackage[left=1in,right=1in,top=1in,bottom=1in]{geometry}

% Increase the line spacing
\renewcommand{\baselinestretch}{1}
% and the spacing between rows in tables
\renewcommand{\arraystretch}{1.5}

% Remove space between items in itemize and enumerate
\usepackage{enumitem}
\setlist{itemsep=0.2em}

% numbering items in reversed orders
\usepackage{etaremune}[start=1]

% Use custom colors
\usepackage[usenames,dvipsnames]{xcolor}

% Set fonts. Requires compilation with xelatex
\usepackage{fontspec}  % required to make older xelatex compile with UTF8

% Set fancy headers
\usepackage{fancyhdr}
\pagestyle{fancy}
\fancyhf{}
\chead{
  \fontsize{10pt}{12pt}\selectfont
  \MyName
  \hspace{0.2cm} -- \hspace{0.2cm}
  \Title
  \hspace{0.2cm} -- \hspace{0.2cm}
  \monthyear\today
}
\rhead{}
\cfoot{\fontsize{10pt}{0}\selectfont \thepage}
\renewcommand{\headrulewidth}{0pt}

% Metadata for the PDF output and control of hyperlinks
\usepackage[colorlinks=true]{hyperref}
\hypersetup{
  pdftitle={\MyName\ - \Title},
  pdfauthor={\MyName},
  linkcolor=blue,
  citecolor=blue,
  filecolor=black,
  urlcolor=MidnightBlue
}


\begin{document}


% No header for the first page
\thispagestyle{empty}


%%%%%%%%%%%%%%%%%%%%%%%%%%%%%%%%%%%%%%%%%%%%%%%%%%%%%%%%%%%%%%%%%%%%%%%%%%%%%%%
% HEADER
\begin{center}

	{\fontsize{36pt}{0}\selectfont \MyName}
	\\[-0.1cm]
	\rule{\textwidth}{0.6pt}
	\\[0.4cm]
	{\fontsize{10pt}{0}\selectfont
		\Affiliation
		\\[0.1cm]
		\Address
		\\[0.1cm]
		Email: \href{mailto:\Email}{\Email}
		\\
		\href{\Website}{Website}
		\, | \,
		\href{\Google}{Google Scholar}
		\, | \,
		\href{\Github}{Github}
		\, | \,
		\href{\ORCID}{ORCID}
	}

\end{center}


%%%%%%%%%%%%%%%%%%%%%%%%%%%%%%%%%%%%%%%%%%%%%%%%%%%%%%%%%%%%%%%%%%%%%%%%%%%%%%%
\section*{Education}
\begin{EntriesTable}

  \Duration{2018}{2022}  &
  \textbf{Ph.D. in Geophysics}, University of Houston, Houston, USA
  \\
  \Duration{2015}{2018}  &
  \textbf{M.S. in Geology}, Northwest University, Xi'an, China
  \\
  \Duration{2011}{2015}  &
  \textbf{B.S. in Geophysics}, China University of Geosciences, Beijing, China

\end{EntriesTable}


%%%%%%%%%%%%%%%%%%%%%%%%%%%%%%%%%%%%%%%%%%%%%%%%%%%%%%%%%%%%%%%%%%%%%%%%%%%%%%%
\section*{Professional Appointments}
\begin{EntriesTable}
	
	\Duration{01/2025}{01/2027}  &
	\textbf{Postdoctoral Research Fellow in MDRU}, University of British Columbia. \textbf{Research directions:} orebody knowledge, mining informatics, geometallurgy
	\\
	\Duration{01/2023}{01/2025}  &
	\textbf{Postdoctoral Research Fellow in Mineral-X}, Stanford University. \textbf{Research directions:} geostatistics, uncertainty quantification, critical mineral exploration

 \end{EntriesTable}



%%%%%%%%%%%%%%%%%%%%%%%%%%%%%%%%%%%%%%%%%%%%%%%%%%%%%%%%%%%%%%%%%%%%%%%%%%%%%%%
\section*{Research Interests}
\begin{itemize}

	\item \textbf{Detect, characterize, and monitor subsurface natural resources}
	    \begin{itemize}
		\item Critical minerals, freshwater, hydrocarbon, and geothermal energy
		\item Subsurface model construction using multiple geoscientific observations (e.g., geophysics, geochemistry, geology, rock physics, etc)
		\end{itemize}

	\item \textbf{Decision-driven Earth Sciences}
 		\begin{itemize}
		\item Bayes’ theorem and uncertainty quantification
		\item Data assimilation and data fusion
		\item Decision theory, reinforcement learning, and AI
		\end{itemize}
		
\end{itemize}



%%%%%%%%%%%%%%%%%%%%%%%%%%%%%%%%%%%%%%%%%%%%%%%%%%%%%%%%%%%%%%%%%%%%%%%%%%%%%%%
\section*{Professional Service \& Outreach}

\subsection*{Editorial service}
\begin{EntriesTable}
	\Duration{2024}{present}  &
	Guest Associate Editor for GEOSCIENCES special section: Geophysical Inversion
	\\
	\Duration{2023}{present}  &
	Guest Associate Editor for GEOPHYSICS special section: Frontiers in Electromagnetic Geophysics
\end{EntriesTable}


\subsection*{Peer-reviewer}
\begin{EntriesTable}
	\Duration{2024}{present}  &
	Ore Geology Reviews
	\\
	\Duration{2024}{present}  &
	Surveys in Geophysics
	\\
	\Duration{2024}{present}  &
	Near Surface Geophysics
	\\
	\Duration{2023}{present}  &
	Geophysical Prospecting
	\\
	\Duration{2023}{present}  &
	Solid Earth
	\\
	\Duration{2022}{present}  &
	Geocarto International
	\\
	\Duration{2021}{present}  &
	SEG Conference Proceeding
	\\
	\Duration{2021}{present}  &
	Geophysics
	\\
	\Duration{2021}{present}  &
	Geophysical Journal International
	\\
	\Duration{2021}{present}  &
	IEEE Transactions on Geoscience and Remote Sensing 
	\\
	\Duration{2021}{present}  &
	Acta Geophysica
	
\end{EntriesTable}


\subsection*{Conference organizations}
\begin{EntriesTable}
	\Duration{2024}{present}  &
	Key Technical Contact of SEG Mining \& Mineral Exploration (MME) Committee 
	\\
	\Duration{2023}{2025} &
	Society of Exploration Geophysicists (SEG) Research Committee Early-career (RCEC) subcommittee
	\\
	\Year{2024} &
	Session Chair for GM P2: Applications in Processing and Interpretation at IMAGE Annual Meeting, Houston, Texas, USA
	\\
	\Year{2024} &
	Session Chair for EM P2: Inversion at IMAGE Annual Meeting, Houston, Texas, USA
	\\
	\Year{2023} &
	Session Co-convener: Advancing Mineral Exploration and Responsible Mining for Energy Transitions, AGU, San Franciscon, California, USA
	\\
	\Year{2023} &
	Session Chair for MME 1: Mineral Exploration: Geophysics 1 at IMAGE Annual Meeting, Houston, Texas, USA
	\\
	\Year{2022} &
	Session Chair for GM 1: Inversion Insights at IMAGE Annual Meeting, Houston, Texas, USA
	\\
	\Year{2021} &
	Session Chair for MG P1: New Methods and Case Histories 1 at IMAGE Annual Meeting (SEG and AAPG joint annual conference), Denver, Colorado, USA
\end{EntriesTable}


\subsection*{Affiliations}
\begin{EntriesTable}
	\Duration{2024}{Present}  &
	Geoscience and Remote Sensing Society (GRSS)
	\\
	\Duration{2024}{Present}  &
	Institute of Electrical and Electronics Engineers (IEEE)
	\\
	\Duration{2022}{Present}  &
	European Geosciences Union (EGU)
	\\
	\Duration{2021}{Present}  &
	Geophysical Society of Houston (GSH)
	\\
	\Duration{2020}{Present}  &
	American Geophysical Union (AGU), European Association of Geoscientists \& Engineers (EAGE)
	\\
	\Duration{2018}{Present}  &
	Society of Exploration Geophysicists (SEG)
\end{EntriesTable}


%%%%%%%%%%%%%%%%%%%%%%%%%%%%%%%%%%%%%%%%%%%%%%%%%%%%%%%%%%%%%%%%%%%%%%%%%%%%%%%
\section*{Invited Talks}
\begin{EntriesTable}
		
	\Year{09/2024} &
	\Wei, AI and ML for mineral exploration discussion panel. Houston, USA. \emph{IMAGE Near Surface Geophysics Technical Sessions}. 
	\\
	
	\Year{09/2024} &
	\Wei, Yin, Z. and Caers, J., A New Role for Geophysics in Mineral Exploration: Falsification of Geological Hypotheses. Houston, USA. \emph{IMAGE Post-Convention Workshop}. 
	\\
	
	\Year{09/2024} &
	\Wei, Building on Dr. Talwani’s foundations: New directions in3D geophysical geometry inversion techniques. Houston, USA. \emph{IMAGE Post-Convention Workshop W11: Manik Talwani Memorial Workshop}. 
	\\
	
	\Year{09/2023} &
	\Wei, Yin, Z., Scheidt, C., Darnell, K., Wang, L. and Caers, J., Uncertainty quantification of the stratigraphic model conditioned on airborne geophysics, geochemistry, and drillholes. Houston, USA. \emph{IMAGE Post-Convention Workshop W7}. 
	\\
	
	\Year{07/2023} &
	Sun, J., and \Wei, Building probabilistic quasi-geology models and mapping mineral resources using joint inversion and geology differentiation. Berlin, Germany. \DOI{10.57757/IUGG23-4333}. \emph{IUGG}. 
	\\
	
	\Year{09/2022} &
	\Wei, Sun, J. and Sen, M., A Bayesian framework for uncertainty quantification of salt body shapes using gravity data. Houston, USA.  \emph{Geophysical Society of Houston}. 
	\\
	
	\Year{11/2021} &
	\WeiSun, Build probabilistic quasi-geology models based on multiple airborne geophysical data and sparse joint inversions. Online. \emph{Geophysical Society of Houston}. 
	\\
	
	\Year{09/2021} &
	\WeiSun, From deterministic to probabilistic geoscience modeling: analyzing uncertainties of geophysical inversions and constructing probabilistic subsurface models conditioned on petrophysical measurements. Online. \emph{SimPEG monthly seminar}. 
	
\end{EntriesTable}



%%%%%%%%%%%%%%%%%%%%%%%%%%%%%%%%%%%%%%%%%%%%%%%%%%%%%%%%%%%%%%%%%%%%%%%%%%%%%%%
\section*{Awards \& Honors}
\begin{EntriesTable}
	
	\Year{2022}  &
	Dan E. Wells Outstanding Dissertation Award, University of Houston (\href{https://uh.edu/nsm/earth-atmospheric/news-events/stories/2022/1219-dissertation-award.php}{link})
	\\
	\Year{2022}  &
	The Innovation Prize in Frank Arnott - Next Generation Explorers Award (\$3,000 CAD)
	\\
	\Year{2022}  &
	SEG Lucien LaCoste Scholarship (\$5,305.12)
	\\
	\Year{2022}  &
	Outstanding Graduate Work in Geophysics, University of Houston (\$1,250)
	\\
	\Year{2022}  &
	The Best Paper in the Mining Sessions at 2021 IMAGE Annual Meeting, Denver, USA (co-author)
	\\
	\Year{2022}  &
	The Best Student Paper in the Mining Sessions at 2021 IMAGE Annual Meeting, Denver, USA
	\\
	\Year{2021}  &
	Student Travel Award, University of Houston, Houston
	\\
	\Year{2021}  &
	SEG Technical Program Registration Grant
	\\
	\Year{2021}  &
	SEG John R. Butler Jr. Scholarship (\$510.86)
	\\
	\Year{2021} &
	The Best Poster in the Mining Sessions at 2020 SEG Annual Meeting, Online
	\\
	\Duration{2020}{2021} &
	Student Research Funding Program (independent of advisor) from EAS Department at University of Houston
	\\
	\Duration{2020}{2021}  &
	Outstanding Academic Achievement, University of Houston (\$700\times2)
	\\
	\Duration{2016}{2018}  &
	The First Prize Scholarship, Northwest University, Xi'an, China (\times3)
	\\
	\Year{2015}  &
	The Best Bachelor Thesis, China University of Geosciences, Beijing, China
	\\
	\Year{2013}  &
	The Second Prize Scholarship, China University of Geosciences, Beijing, China
	
\end{EntriesTable}



%%%%%%%%%%%%%%%%%%%%%%%%%%%%%%%%%%%%%%%%%%%%%%%%%%%%%%%%%%%%%%%%%%%%%%%%%%%%%%%
\section*{Grant}
\begin{EntriesTable}
	
	\Duration{2020}{2021}  &
	\$1,000; Proposal:“Uncertainty Analysis of Geophysical Inversions Conditioned on Spatial Distributions of Geologic Units”; Student Research Funding Program (independent of advisor) from EAS Department at University of Houston; PI: Xiaolong Wei
	
\end{EntriesTable}


%%%%%%%%%%%%%%%%%%%%%%%%%%%%%%%%%%%%%%%%%%%%%%%%%%%%%%%%%%%%%%%%%%%%%%%%%%%%%%%
\section*{Teaching Experience}
\begin{EntriesTable}
	
	\Year{2020}  &
	GEOL7330: Potential Field Methods of Geophysical Exploration (graduate core course), \textbf{Guest Lecturer}. \emph{University of Houston}.
	\\
	\Year{2019}  &
	GEOL4355: Geophysical Field Camp, \textbf{Teaching Assistant}. \emph{University of Houston}.
	
\end{EntriesTable}


%%%%%%%%%%%%%%%%%%%%%%%%%%%%%%%%%%%%%%%%%%%%%%%%%%%%%%%%%%%%%%%%%%%%%%%%%%%%%%%
\section*{Mentoring Experience}
\begin{EntriesTable}
	
	\Duration{2024}{present}  &
	Abdelaziz Zine: PhD visiting student from Mohammed VI Polytechnic University (UM6P) to Stanford University. \textbf{Project title:} Applied geometallurgy to strategic metals ore bodies for a resilient mine operation. \textbf{Co-supervisors:} Jef Caers and Abdellatif Elghali. \textbf{Student Evaluation:} Working under Dr. Xiaolong Wei's supervision at Stanford was transformative for my PhD journey. His deep insights into advanced methodologies like MCMC, level set methods, and simulation significantly enhanced my technical expertise. Beyond his academic guidance, Dr. Wei’s mentorship was invaluable—he generously shared career advice, welcomed my questions, and provided unwavering support, which profoundly shaped both my academic and professional path.
	
\end{EntriesTable}



%%%%%%%%%%%%%%%%%%%%%%%%%%%%%%%%%%%%%%%%%%%%%%%%%%%%%%%%%%%%%%%%%%%%%%%%%%%%%%%
\section*{Publications}
\subsection*{Peer-reviewed}
\begin{etaremune}	
	\item
	\Wei, Yin, Z. and Caers, J., 2024. Falsification of geological hypotheses using drillholes and geophysics. \emph{Surveys in Geophysics} (submitted)	
	
	\item
	\Wei, Yin, Z., Bonner, W. and Caers, J., 2024. Knowledge-driven stochastic modeling of geological geometry features conditioned on drillholes and outcrop contacts. \emph{Computers and Geosciences} (\Review)
	
	\item 
	\Wei, Yin, Z., Schedit, C., Darnell, K., Wang, L. ad Caers, J., 2024. Constructing priors for geophysical inversions constrained by surface and borehole geochemistry \emph{Surveys in Geophysics}. \DOI{10.1007/s10712-024-09843-x}.
	
	\item
	Guo, R., Zhou, H., \Wei, Lin, Z., Li, M., Eldar, Y.C., Yang, F., Xu, S. and Abubakar, A., 2024. Deep joint inversion of multiple geophysical data with U-net reparameterization. \emph{Geophysics}, 90(3), pp.1-68. \DOI{10.1190/geo2024-0210.1}. 
	
	\item
	\Wei, Sun, J. and Sen, M., 2024. 3D Monte Carlo geometry inversion using gravity data. \emph{Geophysics}, 89(3), pp.1-62. \DOI{10.1190/geo2023-0498.1}.
	
	\item
	\Wei, Sun, J. and Sen, M., 2024. Reconstruction of multiple target bodies using trans-dimensional Bayesian inversion with different constraints. \emph{IEEE TGRS}, vol. 62, pp. 1-16. \DOI{10.1109/TGRS.2024.3382106}. 
	
	\item
	Li L., Xiao E., \Wei, Qiu N., Latif K., Guo J. and Sun B., 2023. Crustal Imaging across the Princess Elizabeth Land, East Antarctica from 2D Gravity and Magnetic Inversions. \emph{Remote Sensing}, 15(23):5523. \DOI{10.3390/rs15235523}.
	
	\item
	Hu, Y., \Wei, Wu, X., Sun, J., Chen, J., Huang, Y. and Chen, J., 2023. 3D cooperative inversion of airborne magnetic and gravity gradient data using deep learning techniques. \emph{Geophysics}, 89(1), pp.WB67-WB79. \DOI{10.1190/geo2023-0225.1}.
	
	\item
	Hu, Y., \Wei, Wu, X., Sun, J., Chen, J., Huang, Y. and Chen, J., 2023. A deep learning enhanced framework for multi-physics joint inversion. \emph{Geophysics}, 88(1), pp.1-70. \DOI{10.1190/geo2021-0589.1}.
	
	\item
	\Wei, Sun, J. and Sen, M., 2023. Quantifying uncertainty of salt body shapes recovered from gravity data using trans-dimensional Markov chain Monte Carlo sampling. \emph{Geophysical Journal International}, 232(3), pp.1957-1978. \DOI{10.1093/gji/ggac430}.

	\item
	\Wei, Li, K. and Sun, J., 2022. Mapping critical mineral resources using airborne geophysics, 3D joint inversion and geology differentiation: A case study of a buried niobium deposit in the Elk Creek carbonatite, Nebraska, USA. \emph{Geophysical Prospecting}, 71(Special Issue: Mineral Exploration and Mining Geophysics), pp.1247-1266. \DOI{10.1111/1365-2478.13280}.

	\item
	\WeiSun, 2022. 3D probabilistic geology differentiation based on airborne geophysics, mixed Lp norm joint inversion and petrophysical measurements. \emph{Geophysics}, 87(4), pp.1-67.
	\DOI{10.1190/geo2021-0833.1}. \textbf{Nominated by editors to be highlighted in Geophysics Bright Spots in The Leading Edge }(\href{https://library.seg.org/doi/epub/10.1190/tle41100730.1}{link}).
	
	\item
	\WeiSun, 2021. Uncertainty analysis of 3D potential-field deterministic inversion using mixed L p norms. \emph{Geophysics}, 86(6), pp.G133-G158.
	\DOI{10.1190/geo2020-0672.1}.

	\item
	Sun, J. and \Wei, 2020. Recovering sparse models in 3D potential‐field inversion without bound dependence or staircasing problems using a mixed Lp‐norm regularization. \emph{Geophysical Prospecting}, 69(4), pp.901-910.
	\DOI{10.1111/1365-2478.13063}.

	\item
	Sun, J., Melo, A., Kim, J.D. and \Wei, 2020. Unveiling the 3D undercover structure of a Precambrian intrusive complex by integrating airborne magnetic and gravity gradient data into 3D quasi-geology model building. \emph{Interpretation}, 8(4), pp.1-50.
	\DOI{10.1190/INT-2019-0273.1}.

\end{etaremune}


%\subsection*{In preparation}
%\begin{etaremune}
%	\item
%	\Wei, Yin, Z., Bonner, W. and Caers, J., 2024. Knowledge-driven stochastic modeling of geological geometry features conditioned on drillholes and outcrop contacts. \emph{Computers and Geosciences} (\Review)
%
%	\item
%	\Wei, Yin, Z. and Caers, J., 2024. Falsification of geological hypotheses]{Falsification of geological hypotheses using drillholes and geophysics. \emph{Surveys in Geophysics} (\Review)
%\end{etaremune}


\subsection*{Conference proceedings}
\begin{etaremune}	
	\item 	
	\Wei, Yin, Z. and Caers, J., 2024, August. Falsification of magmatic intrusion models using outcrops, drillholes, and geophysics. In \emph{International Workshop on Gravity, Electrical \& Magnetic Methods and Their Applications}, (pp. 364-367). \DOI{doi.org/10.1190/GEM2024-091.1}.

	\item 	
	Sun, J. and \Wei, 2024, August. Quantifying uncertainty in 3D geophysical inverse problems: Advancing from deterministic to Bayesian and deep generative models. In \emph{International Workshop on Gravity, Electrical \& Magnetic Methods and Their Applications}, (pp. 335-338). \DOI{10.1190/GEM2024-084.1}.

	\item 
	Guo, R., Zhou, H., \Wei, Lin, Z., Li, M., Eldar, Y., Yang, F., Xu, S. and Abubakar, A., 2024, August. Deep joint inversion of electromagnetic, seismic, and gravity data. In \emph{International Workshop on Gravity, Electrical \& Magnetic Methods and Their Applications}, (pp. 360-363). \DOI{10.1190/GEM2024-090.1}.
	
	\item 
	Yang, C., \Wei, Liu, B., Sun, G., Dong, J.E., Sun, L., Tang, X., Li, B. and Ye, G., 2024, August. Targeting potential mineral deposits via uncertainty analysis of magnetic inversions. In \emph{International Workshop on Gravity, Electrical \& Magnetic Methods and Their Applications}, (pp. 224-227). \DOI{10.1190/GEM2024-056.1}.
	
	\item 
	Hu, Y., Wu, X., Sun, J., Huang, Y., Chen, J. and \Wei, 2024, July. Deep Learning Framework for Multi-Physics Joint Inversion and its Application in the Decorah Area. In \emph{IGARSS 2024-2024 IEEE International Geoscience and Remote Sensing Symposium} (pp. 6933-6937). IEEE. \DOI{10.1109/IGARSS53475.2024.10640431}.
	
	\item 
	Hu, Y., \Wei, Wu, X., Sun, J., Huang, Y. and Chen, J., 2023, August. 3D Joint Inversion of Multi-physics Data Using Deep Learning Techniques. In \emph{2023 XXXVth General Assembly and Scientific Symposium of the International Union of Radio Science} (URSI GASS) (pp. 1-4). IEEE. \DOI{10.23919/URSIGASS57860.2023.10265612}.
	
	\item
	\Wei, Sun, J. and Sen, M., 2022. Trans-dimensional Bayesian gravity inversion and uncertainty analysis for salt reconstruction. In \emph{IMAGE Technical Program Expanded Abstracts 2022}.  \DOI{10.1190/image2022-3746659.1}.

	\item
	\WeiSun, 2021. 3D probabilistic geology differentiation using mixed L p norm joint inversion constrained by petrophysical information. In \emph{IMAGE Technical Program Expanded Abstracts 2021}. \DOI{10.1190/segam2021-3586619.1}. \textbf{Best Student Paper in the Mining Sessions}.

	\item
	\WeiSun, 2021. Uncertainty analysis of 3D geophysical inversion using airborne gravity gradient data conditioned on rock sample measurements. In \emph{IMAGE Technical Program Expanded Abstracts 2021}. \DOI{10.1190/segam2021-3586552.1}.

	\item
	Hu, Y., \Wei, Wu, X., Sun, J., Chen, J., Chen, J., Huang, Y., 2021. Deep learning-enhanced multiphysics joint inversion. In \emph{IMAGE Technical Program Expanded Abstracts 2021}. \DOI{10.1190/segam2021-3583667.1}.

	\item
	Li, K., \Wei, Sun, J., 2021. Geophysical characterization of a buried niobium and rare earth element deposit using 3D joint inversion and geology differentiation: A case study on the Elk Creek carbonatite2021. In \emph{IMAGE Technical Program Expanded Abstracts 2021}. \DOI{10.1190/segam2021-3585069.1}. \textbf{Best Paper in the Mining Sessions}.

	\item
	\WeiSun, 2020. Uncertainty analysis of joint inversion using mixed Lp-norm regularization. In \emph{SEG Technical Program Expanded Abstracts 2020} (pp. 925-929). Society of Exploration Geophysicists. \DOI{10.1190/segam2020-3428359.1}.

	\item
	\WeiSun, 2020. Quantifying uncertainties of deterministic geophysical inversions using mixed Lp norms. In \emph{SEG Technical Program Expanded Abstracts 2020} (pp. 1404-1408). Society of Exploration Geophysicists. \DOI{10.1190/segam2020-3420227.1}. \textbf{Best Poster in the Mining Sessions}.

	\item
	Sun, J., Melo, A., Deok Kim, J. and \Wei, 2020. Characterizing a Precambrian intrusive complex by integrating potential field data into 3D quasi-geology model building. In \emph{SEG Technical Program Expanded Abstracts 2020} (pp. 1374-1378). Society of Exploration Geophysicists. \DOI{10.1190/segam2020-3428385.1}.

\end{etaremune}


\subsection*{Conference abstracts}
\begin{etaremune}
	\item 	
	\Wei, Yin, Z. and Caers, J., 2024, December. Falsifying geological hypotheses using geophysics. In \emph{AGU Fall Meeting Abstracts}.
	
	\item 
	\Wei, Yin, Z., Schedit, C., Darnell, K., Wang, L. and Caers, J., 2023, December. Quantifying uncertainty for sediment-hosted mineral deposits using multiple geoscientific observations and Bayesian evidential learning. In \emph{AGU Fall Meeting Abstracts}.
	
	\item
	Sun, J., and \Wei, 2023, August. Mapping critical mineral resources using multiphysics inversion. In \emph{IMAGE Technical Program Abstracts 2023}. 
	
	\item
	Hu, Y., \Wei, Wu, X., Sun, J., Chen, J., Chen, J., Huang, Y., 2023, August. Deep learning enhanced joint inversion for mineral exploration using airborne geophysics: Application in Decorah area. In \emph{IMAGE Technical Program Abstracts 2023}.
	
	\item
	\Wei, Sun, J. and Sen, M., 2023, August. 3D trans-dimensional Monte Carlo geometry inversion and uncertainty quantification using gravity data. In \emph{IMAGE Technical Program Abstracts 2023}. 
	
	\item
	Sun, J., \Wei \, and Sen, M., 2023, August. Uncertainty quantification of anomalous body shapes using potential field data in a trans-dimensional Bayesian framework, \emph{XXVIII General Assembly of the International Union of Geodesy and Geophysics} (IUGG) (Berlin 2023).  \DOI{10.57757/IUGG23-4343}.
	
	\item
	\Wei, Sun, J. and Sen, M., 2022, December. A Bayesian framework for uncertainty analysis of anomalous body shapes using gravity data. In \emph{AGU Fall Meeting Abstracts} (\href{https://ui.adsabs.harvard.edu/abs/2022AGUFMNG35B0469W/abstract}{Vol. 2022, pp. NG35B-0469}).

	\item
	\WeiSun, 2021, December. Building 3D probabilistic geology differentiation models using mixed Lp norm joint inversion, airborne geophysics and petrophysical information. In \emph{AGU Fall Meeting Abstracts} (\href{https://ui.adsabs.harvard.edu/abs/2021AGUFMNG25A0485W/abstract}{Vol. 2021, pp. NG25A-0485}).

	\item
	\WeiSun, 2021, December. Analyzing uncertainty of 3D inversion using airborne geophysical data conditioned on petrophysical measurements. In \emph{AGU Fall Meeting Abstracts} (\href{https://ui.adsabs.harvard.edu/abs/2021AGUFMNS35C0373W/abstract}{Vol. 2021, pp. NS35C-0373}).

	\item
	Li, K., \Wei, Sun, J., 2021, December. Characterizing a buried niobium deposit using airborne geophysics, joint inversion, and geology differentiation. In \emph{AGU Fall Meeting Abstracts} (\href{https://ui.adsabs.harvard.edu/abs/2021AGUFMNS24A..05L/abstract}{Vol. 2021, pp. NS24A-05}).

\end{etaremune}


\subsection*{Open code and data}
\begin{etaremune}
	\item
	\WeiSun, 2021. Contributor of the joint inversion code to the open source package SimPEG

	\item
	\WeiSun, 2021. Joint inversion of gravity gradient and magnetic data using mixed Lp norm regularization (1.0). \emph{Zenodo}. \DOI{10.5281/zenodo.5774303}.

	\item
	\WeiSun, 2021. Interactive geology differentiation and 3D visualization of geological units (1.0). \emph{Zenodo}. \DOI{10.5281/zenodo.5774309}.

	\item
	Sun, J., and \Wei, 2020. Solving the bound dependence and staircasing problems in 3D potential-field sparse inversions using a mixed Lp-norm regularization (1.0). \emph{Zenodo}. \DOI{10.5281/zenodo.4057134}.

\end{etaremune}


%%%%%%%%%%%%%%%%%%%%%%%%%%%%%%%%%%%%%%%%%%%%%%%%%%%%%%%%%%%%%%%%%%%%%%%%%%%%%%%
\section*{Certifications}
\begin{EntriesTable}

%	\Year{2022} &
%	Convolutional Neural Networks course given by Dr. Andrew Ng through Coursera, Inc.
%	\\
	\Year{2022} &
	Remote pilot for the small unmanned aircraft system issued by Federal Aviation Administration
	\\
	\Year{2021} &
	FAA Part 107 Knowledge Test Prep for Drone Pilot on Udemy, Inc.
	\\
	\Year{2021} &
	ISInProG@Lario - 2021 International School on Inverse Problems in Geophysics on the shore of the Lario Lake
%	\\
%	\Year{2021} &
%	Magnetotellurics (MT) short course given by Dr. Alan G. Jones
%	\\
%	\Year{2018}{}  &
%	Machine Learning course given by Dr. Andrew Ng through Coursera, Inc.

\end{EntriesTable}

\end{document}
